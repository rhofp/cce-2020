\section{Background}

All recent efforts to work with object recognition successfully can be 
encompassed in two large model families. At first we have algorithms 
related to convolutional neural networks (CNN), and  in second place  the YOLO 
(You Only Look Once) approach.
% Todos los esfuerzos recientes para el reconocimiento de objetos de manera 
% satisfactoria se pueden englobar en dos grandes familias de modelos. En primer 
% lugar se tiene la utilización de algoritmos relacionados con redes neuronales 
% convolucionales (CNN), y en segundo lugar se tiene el enfoque YOLO (You Only 
% Look Once).

Models based on convolutional neural networks, as the name implies, use neural 
networks with an extra layer called the convolution layer, which has the 
characteristic of choosing or detecting patterns and making sense of them. 
Such pattern detection is what makes CNN useful for image analysis. CNN works 
through filters, which allow detecting patterns such as circles, edges, squares, 
lines, among others. The famous CNN-based models are: R-CNN, Fast R-CNN, and 
Faster R-CNN.
% Los modelos basados en redes neuronales convolucionales, como su nombre lo 
% indica, utilizan redes neuronales con una capa extra llamada capa de 
% convolución, la cual tiene la característica de escoger o detectar patrones 
% y darles sentido. Dicha detección de patrones es lo que hace que CNN sea útil 
% para el análisis de imágenes. Las CNN trabajan a través de filtros, los cuales 
% permiten detectar patrones como círculos, bordes, cuadrados, líneas, entre 
% otros. Los modelos famosos basados en CNN son: R-CNN, Fast R-CNN y Faster 
% R-CNN.

On the other side, YOLO, currently supported by DarkNet, is a real-time object 
detection system. It works by dividing the image into cells, where each cell is 
tasked with predicting a bounding box that involves 3 elements: X and Y 
coordinates, width and height dimensions, and a confidence value. With these 
elements it is possible to try to predict the objects found in the image, 
involving a single trained neural network, hence the name YOLO. In recent years, 
improvements have been made to this model in order to make it faster and more 
reliable. These improvements are known YOLO versions 2 and 3.
% Por otra parte, YOLO, actualmente soportado por DarkNet, es un sistema de 
% detección de objetos en tiempo real. Funciona dividiendo la imagen en celdas, 
% donde cada celda tiene la tarea de predecir un cuadro delimitador que 
% involucra 3 elementos: coordenadas X y Y, las dimensiones ancho y alto, y un 
% valor de confianza. Con estos elementos se puede intentar predecir los objetos 
% que se encuentran en la imagen, involucrando a una sola red neuronal 
% entrenada, de ahí el nombre de YOLO. En los últimos años se le han hecho 
% mejoras a este modelo con la finalidad de hacerlo más rápido y confiable, 
% dichas mejoras llevan por nombre YOLO v2 y YOLO v3.

Knowledge graphs started to be known in 2012, thanks to 
Google, who decided to add a semantic improvement to its search 
engine, called “The Knowledge Graph”. Its purpose was to answer 
questions for users through analysis of what words actually mean in a query, 
rather than analyzing character strings. So, today it is about things and not 
strings \cite{Barnard}. Thas why many big companies began to create their own 
knowledge graphs, such as Amazon, Microsoft, Yahoo, Facebook, among others; 
that drive semantic searches and enable better data handling, with better 
data processing and smarter outputs (deliveries).
% Con respecto a los grafos de conocimiento, este campo empezó a ser conocido 
% en 2012, gracias que Google decidió agregar una mejora semántica en su motor 
% de búsqueda, llamado “The Knowledge Graph”. El objetivo de este fue responder 
% preguntas para los usuarios a través del análisis de lo que realmente 
% significan las palabras en una consulta, en lugar de analizar cadenas de 
% caracteres. Así, hoy en día se trata de cosas y no de cadenas \cite{Barnard}.
% A partir de esto, muchas compañías comenzaron a crear sus propios grafos de 
% conocimientos, como lo Amazon, Microsoft, Yahoo, Facebook, entre otras; que 
% impulsan las búsquedas semánticas y permiten un mejor manejo de los datos, 
% con un mejor procesamiento de datos y salidas (entregas) más inteligentes.

Currently, there are different knowledge graphs that allow companies to 
create their own knowledge network, in order to improve their domain and 
operation. A clear example of success in the use of knowledge graphs at the 
Prado Museum in Spain, where the information systems of archives, libraries, 
and collections were exchanged for a knowledge graph. This knowledge graph is 
a system of representation of contents and digital resources of facts related 
to authors, works of art, their contents, themes, periods, and styles, as 
well as any object potentially related to them \cite{Museo del Prado}.
% En la actualidad, existen diferentes grafos de conocimiento que permiten que 
% las empresas crear su propia red de conocimiento, con el fin de mejorar el 
% dominio y funcionamiento de éstas. Un claro ejemplo de éxito en el uso de 
% grafos de conocimiento en el Museo del Prado de España, donde se cambió los 
% sistemas de información de archivos, bibliotecas y colecciones por un grafo 
% de conocimiento. Este grafo de conocimiento es un sistema de representación de 
% contenidos y recursos digitales de hechos relacionados con los autores, las 
% obras de arte, sus contenidos, temas, épocas y estilos, así como cualquier 
% objeto potencialmente relacionado con éstos \cite{Museo del Prado}.

In this sense, bringing together two areas of knowledge, object detection, and 
knowledge graphs, to achieve a common objective, such as detecting an object 
with greater precision and using fewer computational resources is one of the 
great challenges posed in the current decade. Thus, the works and 
publications on this topic are scarce.
% En este sentido, juntar dos área del conocimiento, la detección de objetos y 
% los grafos de conocimiento, para lograr un objetivo en común, como la 
% detección de un objeto con mayor precisión y usando menos recursos 
% computacionales es uno de los grandes retos planteados en la década actual. 
% Por lo que, los trabajos y publicaciones sobre dicho tema son escasos.

Therefore, the starting point of this paper is based on the object 
characterization and detection concern through knowledge graphs \cite{Fang}, 
which studies computer vision for object recognition, where the purpose is 
to identify a set of regions and thus be able to classify each section with 
labels. It was possible to obtain new images with unobserved contexts in 
previously elaborated algorithms. Managing to demonstrate that these existing 
algorithms could be optimized obtaining a better semantic relationship.
% Por lo tanto, el punto de partida de este artículo nace a partir del interés 
% de la caracterización y detección de objetos mediante los gráficos de 
% conocimiento \cite{Fang}, que estudia la visión por computadora para el 
% reconocimiento de objetos, donde el objetivo fue identificar un conjunto de 
% regiones y así poder clasificar cada sección con etiquetas. Se logró obtener 
% nuevas imágenes con contextos no observados en algoritmos previamente 
% elaborados. Logrando demostrar que se pudieron optimizar estos algoritmos 
% existentes obteniendo una mejor relación semántica.
