\section{Background}

All recent efforts to work with object recognition successfully can be 
encompassed in two large model families. At first we have algorithms 
related to convolutional neural networks (CNN), and  in second place  the YOLO 
(You Only Look Once) approach.

Models based on convolutional neural networks, as the name implies, use neural 
networks with an extra layer called the convolution layer, which has the 
characteristic of choosing or detecting patterns and making sense of them. 
Such pattern detection is what makes CNN useful for image analysis. CNN works 
through filters, which allow detecting patterns such as circles, edges, squares, 
lines, among others. The famous CNN-based models are: R-CNN, Fast R-CNN, and 
Faster R-CNN.

On the other side, YOLO, currently supported by DarkNet, is a real-time object 
detection system. It works by dividing the image into cells, where each cell is 
tasked with predicting a bounding box that involves 3 elements: X and Y 
coordinates, width and height dimensions, and a confidence value. With these 
elements it is possible to try to predict the objects found in the image, 
involving a single trained neural network, hence the name YOLO. In recent years, 
improvements have been made to this model in order to make it faster and more 
reliable. These improvements are known YOLO versions 2 and 3.

Knowledge graphs started to be known in 2012, thanks to 
Google, who decided to add a semantic improvement to its search 
engine, called “The Knowledge Graph”. Its purpose was to answer 
questions for users through analysis of what words actually mean in a query, 
rather than analyzing character strings. So, today it is about things and not 
strings \cite{Barnard}. Thas why many big companies began to create their own 
knowledge graphs, such as Amazon, Microsoft, Yahoo, Facebook, among others; 
that drive semantic searches and enable better data handling, with better 
data processing and smarter outputs (deliveries).

Currently, there are different knowledge graphs that allow companies to 
create their own knowledge network, in order to improve their domain and 
operation. The Prado Museum, in Spain, is a clear success example of knowledge 
graphs use, they changed information systems of archives, libraries, 
and collections into a knowledge graph, which is now  
a system of representation of contents and digital resources of facts related 
to authors, works of art, their contents, themes, periods, and styles, as 
well as any object potentially related to them \cite{Museo del Prado}.

In this sense, bringing together two areas of knowledge, object detection, and 
knowledge graphs, to achieve a common objective, such as detecting an object 
with greater precision and using fewer computational resources is one of the 
great challenges posed in the current decade. Thus, the works and 
publications on this topic are scarce.

Therefore, the starting point of this paper is based on the object 
characterization and detection concern through knowledge graphs \cite{Fang}, 
which studies computer vision for object recognition, where the purpose is 
to identify a set of regions and then be able to classify each section with 
labels. It was possible to obtain new images with unobserved contexts in 
previously elaborated algorithms. Managing to demonstrate that these existing 
algorithms could be optimized obtaining a better semantic relationship.