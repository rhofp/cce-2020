\begin{abstract}
In this paper, we propose a knowledge graph to describe the object semantic 
relationships that there are in an indoor space, such as a bedroom. To do so, 
we base our work on the representation model Object1--Predicate--Object2. In 
the predicate, we consider spatial relationships such as above, below, under, 
on top, next to, in, has, in front, and behind. To fulfill our purpose, it was 
used the Grakn NoSQL database, which allows defining a knowledge graph schema of
type entity-relation. The obtained knowledge graph needs a previously identified 
object in order to start looking for the correct relationships. That object is 
considered as input to the algorithm. The information on the semantic 
relationship has demonstrated its effectiveness in characterizing objects using 
Grakn, through which it was possible to characterize objects and their 
relationships based on the principles of knowledge representation.
\end{abstract}

\begin{IEEEkeywords}
    Knowledge Graphs, Grakn, indoor spaces, objects, semantic relationships.
\end{IEEEkeywords}