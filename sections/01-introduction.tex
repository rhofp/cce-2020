\section{Introduction}
Within the past ten years, object recognition
has gained importance due to the significant development of 
hardware (CPUs and GPUs with higher performance and speed) and thanks to the 
constant progress of computer science (increasingly sophisticated machine 
learning, and deep learning algorithms). Object detection involves two main 
tasks: i) image classification and ii) object localization.

Image classification is intended to predict what kind of objects are in 
an image and object localization is about locating the object 
with a bounding box. In general, object recognition uses artificial 
neural networks, which despite being a powerful tool, they have the disadvantage 
of becoming slow through the large number (millions) of node networks 
communicating with each other. Hence, it is been necessary to propose different 
approaches to reduce processing time, thats why we address knowledge graphs.

A knowledge graph is a set of organized information, where each vertex 
represents specific data which can be accessed through a specific edge. 
On the other hand, \cite{Saorin} points out that a knowledge graph connected 
with a database enriches and increases its
significance, for that reason entities and properties can be defined, thus, 
those properties get classified and model  into our domain of knowledge. 
By having the domain to which it belongs, it allows another graph to connect it 
to another domain model.

The following example illustrates the previous idea: A system understands that 
a human being is an entity of soccer player type, and it is linked with soccer 
sport, which is played with a ball and where there are many competitions. 
In this way, a knowledge graph may have the following content related in a 
semantic way:

\begin{itemize}
\item Data of place, a person or a certain company.
\item Images.
\item Text excerpts.
\item Data with secondary details.
\item References to similar searches.
\end{itemize}

This paper seeks to communicate the use of a knowledge graph as a tool to 
characterize the semantic relationships that exist between objects existing 
in a certain closed environment of human occupation, such as a room in a home, 
office, and other places for resting or working. To achieve this purpose, we use 
a basic model with three components: object1-predicate-object2. 
In predicate we use spatial relations such as: above, below, 
above, beside, in, has, in front and others.

The document is organized as follows, Section 2 presents the background of the 
degrees of knowledge, their applications are discussed and related works are 
presented. Section 3 describes the method established as a proposed solution. 
Section 4 presents the results obtained, based on an example of application in 
a room, and Section 5 summarizes some conclusions and future work.